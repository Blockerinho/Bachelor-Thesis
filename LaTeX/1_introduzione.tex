\chapter{Introduzione}

\section{Introduzione generale}
Nella stesura di questo lavoro di tesi il punto di partenza è stato lo studio della pubblicazione Routing permutations on graphs via matchings” di Noga Alon, F.R.K. Chung e R.L. Graham \cite{14}. In particolare, ci siamo concentrati sulla parte introduttiva relativa al problema delle permutazioni di instradamento su grafi. 

A partire da questo spunto, l’obiettivo principale è stato quello di indagare se e in che modo fosse possibile ricondurre il gioco del 15 al quadro teorico delineato dagli autori, così da ottenere strumenti utili per affrontarne la risoluzione (\ref{modellazione}). Per farlo, è stato necessario innanzitutto definire in modo rigoroso le basi teoriche del problema: dalla formalizzazione delle permutazioni e della loro parità (\ref{modellazione}), alla modellizzazione del puzzle come grafo, fino ai richiami essenziali alla teoria della complessità (\ref{complessita}) e alla caratterizzazione delle classi P e NP. (\ref{complessita})

Su queste fondamenta si è poi sviluppata l’analisi delle strategie risolutive: dagli algoritmi di ricerca euristica (A*, IDA*) alle ottimizzazioni basate su euristiche come la distanza di Manhattan e i conflitti lineari (\ref{codice}), fino a un approccio alternativo basato su Answer Set Programming (\ref{asp}).

\section{Formalizzazione del problema} \cite{14}

Immaginiamo di avere un grafo $G$ e di aver definito i nodi del grafo in modo che ognuno di essi abbia un colore. 
Allo stesso modo supponiamo che ogni nodo del grafo contenga sopra una pedina, la quale assume un colore, che non per forza è uguale a quello che assume il nodo su cui si trova. 
Per definire il problema in modo esatto, definiamo: 
\begin{itemize}
    \item $G=(V,E)$ un grafo connesso con self-loop in cui $|V|=n$;
    \item $\pi(v)$ permutazione finale del nodo $v$ del grafo;
    \item $\pi_0(v)$ permutazione iniziale del nodo $v$ del grafo;
\end{itemize}
Si ottiene, quindi, che $\pi$ rappresenta l'insieme delle permutazioni finali di ogni nodo del grafo e $\pi_0$ rappresenta l'insieme delle permutazioni iniziali dei nodi del grafo. Vogliamo che a ogni passo valga la seguente relazione:
$$
\forall i \forall v (\pi_i(v), \pi_{i+1}(v)) \in E
$$
tenendo conto che $\pi_i(v)$ rappresenta la permutazione del nodo $v$ al tempo $i$ e $i+1$, questo vuol dire che una pedina può solo essere scambiata con una pedina adiacente ad essa, quindi la permutazione del nodo $v$ al tempo $i$ e quella al tempo $i+1$ formano un arco del grafo.

Definiamo: 
\begin{itemize}
    \item $P_v(t)$: posizione della pedina con posizione iniziale $v$ al tempo $t$; 
    \item $P_v(t) = \pi_t(\pi_0^{-1}(v))$; 
    \item dove $\{P_v(t): v \in V \}$ è una permutazione.
\end{itemize} Otteniamo quindi il numero minimo di passi per passare da $\pi_0$ a $Id$:
$$
rt(G,\pi_0)
$$
Di conseguenza definiamo come routing number il valore:
$$
rt(G) = \underset{\pi_0}{\max} \ rt(G, \pi_0)
$$
come il numero massimo di mosse necessarie per riportare ogni pedina nella sua posizione ottimale con la sequenza più breve possibile.