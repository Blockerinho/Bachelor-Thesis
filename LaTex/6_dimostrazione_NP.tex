\chapter{Dimostrazione della NP-completezza della ricerca della sequenza di mosse minore per passare da una configurazione a un altra nel gioco del 15}
 In principio il problema di calcolare la minor sequenza di mosse che servono per passare da una configurazione del gioco del 15 a un altra è stato dimostrato, da parte di Oded Goldreich, essere un problema NP-hard.
 Nella sua dimostrazione Goldreich dimostra tale assunto riducendo il problema di 3-Exact-Cover (3XC), descritto sopra, a una formalizzazione del problema del gioco del 15. ~\cite{2}

La dimostrazione però può essere fatta anche in un secondo modo ovvero considerando i concetti di gruppo e generatore.

\section{Definizione di gruppo}
Un gruppo è una struttura algebrica $(G, \cdot)$ costituita da un insieme $G$ non vuoto munito di un'operazione binaria $\cdot : G \times G \to G$ che soddisfa i seguenti assiomi:
\begin{itemize}
    \item Associatività: Per ogni $a, b, c \in G$, si ha $(a \cdot b) \cdot c = a \cdot (b \cdot c)$
    \item Elemento neutro: Esiste un elemento $e \in G$ tale che per ogni $a \in G$, $e \cdot a = a \cdot e = a$
    \item Elemento inverso: Per ogni $a \in G$, esiste un elemento $a^{-1} \in G$ tale che $a \cdot a^{-1} = a^{-1} \cdot a = e$
\end{itemize}

Nel contesto delle permutazioni, i gruppi emergono dal fatto che ogni permutazione di un insieme finito $X = {1, 2, \ldots, n}$ può essere vista come una funzione biiettiva $\sigma: X \to X$. L'insieme di tutte le permutazioni di $X$, denotato $S_n$, forma un gruppo rispetto alla composizione di funzioni, chiamato gruppo simmetrico di grado $n$.

\section{Generatori di un gruppo}
Sia $(G, \cdot)$ un gruppo e sia $S \subseteq G$ un sottoinsieme. Il sottogruppo generato da $S$, denotato $\langle S \rangle$, è il più piccolo sottogruppo di $G$ che contiene tutti gli elementi di $S$.
Equivalentemente, $\langle S \rangle$ è l'insieme di tutti gli elementi di $G$ che possono essere espressi come prodotto finito di elementi di $S$ e dei loro inversi.

Un sottoinsieme $S \subseteq G$ si dice insieme di generatori di $G$ se $\langle S \rangle = G$, ovvero se ogni elemento di $G$ può essere espresso come combinazione finita di elementi di $S$ e dei loro inversi.

\section{Gruppi, permutazioni e gioco del 15}
Nel nostro caso possiamo formalizzare il gioco del 15 con un gruppo.
Consideriamo: 
\begin{itemize}
    \item un insieme di elementi: nel nostro caso i numeri da 1 a 16;
    \item i generatori come qualsiasi trasposizione di una pedina con la cella vuota (16); 
\end{itemize}

\section{Ricerca del numero minimo di mosse per passare da una permutazione a un altra e teoria dei gruppi}

Il motivo per cui siamo interessati ai gruppi e alla teoria dei gruppi è perché la ricerca del numero minore di mosse che portano da una configurazione del piano di gioco alla soluzione corrisponde al problema di trovare la più breve sequenza generatrice che realizza una permutazione. ~\cite{2} 

Problema che è stato dimostrato essere NP-hard. ~\cite{9} 
Essendo ogni generatore una trasposizione tra una pedina e la cella vuota, questo equivale a cercare il numero minimo di trasposizioni che portano alla soluzione.
Ma essendo ogni trasposizione (mossa) rappresentabile con una permutazione questo equivale a cercare il numero minimo di permutazioni del piano di gioco che portano alla soluzione. 

\section{Grafi di Cayley e problema del calcolo del diametro}
Il nostro problema può essere dimostrato essere NP-hard seguendo anche una seconda strada, ovvero sfruttando il concetto di grafo di Cayley. 

Un grafo di Cayley è una struttura matematica che fornisce una rappresentazione geometrica di un gruppo attraverso la teoria dei grafi. Formalmente, dato un gruppo finito $G$ e un insieme di generatori $S \subseteq G$ tale che $S$ non contenga l'elemento neutro e sia chiuso rispetto all'inverso, il grafo di Cayley contiene un vertice che corrisponde a ogni elemento del gruppo $G$ e un arco tra due vertici $g$ e $h$ se esiste un generatore $s \in S$ tale che $h$ può essere ottenuto da $g$ moltiplicando per $s$. 

Nel nostro caso ogni vertice del grafo corrisponde a una configurazione del puzzle.
Ogni arco collega due configurazioni se è possibile passare dall’una all’altra tramite una mossa elementare.
Così, il grafo di Cayley rappresenta tutte le possibili disposizioni e le transizioni tra di esse.

Il diametro di un grafo connesso $G$ è la massima distanza tra due vertici qualsiasi del grafo:
\[
\text{diam}(G) = \max_{u,v \in V(G)} d(u,v)
\]
dove $d(u,v)$ denota la lunghezza del cammino più breve tra $u$ e $v$.

Nel contesto dei grafi di Cayley, il diametro ha un'interpretazione algebrica particolarmente significativa: rappresenta il numero minimo di elementi dell'insieme generatore $S$ necessari per esprimere qualsiasi elemento del gruppo come prodotto di generatori. Nel puzzle del 15, il diametro del grafo di Cayley corrisponde al numero massimo di mosse minime necessarie per risolvere qualsiasi configurazione. 

Calcolare il diametro di un grafo di Cayley è stato più volte dimostrato essere un problema NP-hard. ~\cite{10}
